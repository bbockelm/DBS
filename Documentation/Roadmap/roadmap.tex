
\documentclass{cmspaper}
\begin{document}

%==============================================================================
% title page for few authors

\begin{titlepage}

% select one of the following and type in the proper number:
%   \cmsnote{2005/000}
  \internalnote{2006/000}
%  \conferencereport{2005/000}
   \date{1 August, 2006}

  \title{A 2006-07 Roadmap for CMS Dataset Bookkeeping System}

  \begin{Authlist}
   Lee Lueking, Andrew Dolgert, Vijay Sekhri, Anzar Afaq, Valentin Kuznetsov, Alessandra Fanfani, Stefano Belforte, Lothar Bauerdick, Ian Fisk, Elizabeth Sexton-Kennedy, Chris Jones, Dan Riley, Stefano Lacaprara, Peter Elmer  \Instfoot{fnal}{FNAL, Batavia, IL, USA}
%    R.~Egeland \Instfoot{minn}{University of Minnesota, Twin Cities, MN, USA}
  \end{Authlist}

% if needed, use the following:
%\collaboration{Flying Saucers Investigation Group}
%\collaboration{CMS collaboration}

%  \Anotfoot{a}{On leave from prison}
%  \Anotfoot{b}{Now at the Moon}

  \begin{abstract}
Mission: To bring together experts in data management, workflow management, CMSSW framework and other CMS areas, to discuss, document and devise a workplan for the dataset bookkeeping system we will follow for the next 6 months. The system we discuss will provide the needed function and features for the testing throughout 2007 and the beginning of datataking in Fall of 2007. Establish the overall vision, requirements, and scope for the project.
 
  \end{abstract} 
  
\end{titlepage}

\setcounter{page}{2}%JPP


%
%==============================================================================

\section{Introduction}

The DBS Workshop held at Cornell July 20-22 of 2006 brought together 18 people experienced in many areas of CMS software and Computing to discuss the uses and plan for the CMS  Dataset Bookeeping System. The DBS project has developed over the last year  a prototype architecture,  schema and API that is sufficient for the CSA06 challenge work, but insufficient to meet CMS's longterm needs. Our mission was to bring together experts in data management, workflow management, CMSSW framework and other CMS areas, to discuss, document and devise a workplan for the dataset bookkeeping system we will follow for the next 6 months. The system we discuss will provide the needed function and features for the testing throughout 2007 and the beginning of datataking in Fall of 2007. Establish the overall vision, requirements, and scope for the project.

\section{DBS Workshop Overview}

\subsection{Goals for Cornell workshop}

Workshop Goals:
\begin{itemize}
\item Agree to a set of common definitions used in describing the DBS system. Deliverable: Glossary document.
\item  Produce a set of use cases we think are representative of the needs of CMS analysis in general, but targeting the DBS specifically. Deliverable: Documented use cases and work user flow diagrams illustrating them.
\item  Explore existing data management and discovery systems employed by HEP experiments and identify features we would like to have for CMS. Deliverable: Document describing key features we may have overlooked in CMS.
\item  Understand the needed relationships between DBS and the other components of DM and WM systems. Deliverable: Concept diagram showing the relationships.
\item  Understand the relationships between DBS and the CMSSW Framework. Deliverables: Document describing this and diagrams as needed.
\item  Understand the relationships between DBS and the many other sources of non-event information in CMS such as trigger, luminosity, run configuration, run quality information, calibration, alignment, ... Deliverables: Concept map showing the relationships.
\item  Understand how DBS will fit into the picture for the various tiers of CMS computing such as, online, Tier 0, Tier 1, Tier 2 and Tier-n. Deliverable: Document the specific and overlapping needs for each tier.
\item  Establish a set of written requirements for the DBS. Deliverable: Document
\item  Review the existing CMS DBS system, its schema and API
\item  Devise (revise) the schema and API to meet the requirements. Deliverable: Updated schema and APIs. Revise existing DBS description document.
\item  Update the design for the overall DBS system, servers, clients, et cetera. Deliverable: Description document and diagrams.
\item  Explore options for data discovery desirables, and techniques for doing them. Deliverable: Document and list of things to try out.
\item  Discuss possible implementation options. Prototype if time permits. Document issues and further work needed to make final choices. Deliverable: Description of options and motivation for any particular choices. Description of additional work needed.
\item  Draft a plan, describe deliverables, estimate work, prioritize, schedule and assign tasks. Deliverable: Description, schedule, assignments, milestone dates.
\end{itemize}

\subsection{Workshop Topics}

\begin{itemize}
\item Review of existing DBS system
\item Glossary of terms
\item Use cases for Data analysis
\item EDM/FW connections to DBS 
\item DBS In the context of DM and WM
\item A look at CLEO, Run II, and BaBaR approaches
\item DLS Interactions with DBS 
\item CRAB Interactions with DBS
\item Monte Carlo Request system present and future 
\item Additional DBS Scenarios 
\item Details of and alternatives to existing prototype 
\item Data discovery needs and approaches
\end{itemize}

\section{Glossary} 
\begin{itemize}
\item Dataset-"any set of 'event collections' that would naturally be
grouped and analysed together as determined by
physics attributes" -ctdr
e.g., common trigger path or MC generator (configuration?)
e.g., "a particular object model representation of those events" such as
RAW, RECO, AOD??
\item Primary Dataset-"data at all levels of processing pertaining to a given
trigger or common MC production criteria" -dm-dbsv0.85
"For Monte Carlo data 'primary datasets' comprise
data generated with the same parameters" -dm-dbsv0.85
Is the set of MC primary datasets bounded? Determinable?
\item Processed Dataset- "a slice of data from a Primary Dataset defined by the
processing history applied to it" -dm-dbs-v0.85
"will correspond to a large production of data with a
single major software release version, but may include
multiple minor versions for small bug fixes and also
may contain the output of multiple processings of
some given input event collection" -dm-dbs-v0.85
��multiple minor versions lost in some versions of the DBS schema
\item Analysis Dataset- "A subset of Event Collections from a given Processed
Dataset" -dm-dbs-v0.85
Adds constraints to processed dataset
Similar to CLEO EventStore ��physics grade--the data physicists should run
on?
Gone from current schema, no mechanism to create one
\item Data Tier- "2.5.1 Data Tiers" -ctdr
DAQ-RAW, RAW, RECO, AOD, TAG, FEVT
"type of content resulting from a degree of processing
applied to the data" -dm-dbs-v0.85
What about RECO and AOD?
Actually defined by the objects contained?
Then data tier is an attribute of a file or event collection, not processing?
What about data tier evolution? AOD2, AOD3, ...?
Or finer grained processing path (not in framework provenance)?
\item Event Collection-The smallest unit that a user is able to select through
the dataset bookkeeping system" -ctdr
e.g., event data from a particular trigger selection from one given run
"a single event collection will give access to data from
exactly one data tier" -dm-dbs-v0.85
Currently supporting multiple data tiers, e.g., FEVT
Relationship to file one-to-many or many-to-many
many-to-many for old EDM?
\item File Block- "a set of files likely to be accessed together" -dm-dbsv0.85
data management packaging unit
"will in general correspond to the same Primary
Dataset, Data Tier and Processed Dataset" -dm-dbsv0.85
incorrectly reflected in schema?
\item non-event data-Non-event data: detector configuration, conditions, ...
Mostly time domain: run, luminosity interval?
DBS currently knows nothing about this
so we told Drew not to work on them
Is this in scope for DBS? If not, what should answer these queries?
\end{itemize}


\section{DBS interaction with other DM, WM areas, and CMS at large}

\subsection{DLS}
The Data Location Service (DLS) provides the means to locate replicas of data in the distributed computing system
it maps file-blocks to storage elements (SEs)
provide only names of sites hosting the data and not the physical location of constituent files at the sites
few attributes (custodial  replica = considered a 
   permanent copy at a site)

 Interactions with DLS
Insert file-blocks produced at a site (MC Production)
Insert file-blocks upon data replication (PhEDEx)
Query to locate file-blocks (CRAB, MC Production)


DLS twiki: https://twiki.cern.ch/twiki/bin/view/CMS/DLS

Fileblock name
DBS knows how dataset are organized in term of fileblocks
DLS knows fileblocks locations
 Fileblock name needs to be consistent across DBS and DLS
Local scopes DBS/DLS upload into global DBS/DLS (clashes on fileblock name?)
Data discovery and location: combine DBS and DLS information

\subsection{PhEDEx}

\subsection{DBS Context Diagram}
Fig.~\ref{fig:dbs-context-diagram}
% I have a png file for this, need to convert to eps and 
% have some technical problem on my new laptop.
\begin{figure}[hbtp]
  \begin{center}
    \resizebox{15cm}{!}{\includegraphics{dbs-context-map.eps}}
    \caption{Concept diagram for DBS within CMS.}
    \label{fig:dbs-context-diagram}
  \end{center}
\end{figure} 

\section{Tasks and Use Cases from the  Workshop}


List from the WS

These will be reviewed, sorted, and assigned for further work. Many people volunteered (or were volunteered ) to help.


   1.  (use case, fermilab)Need to develop use case for publishing locally analyzed data to global scope - Fermilab

   2. (dbs)Fix a subset of a dataset due to problem found later, e.g. sqrt (-1) (changing does not affect the physics when it does not happen), calibration change. These are cases where only selected parts of the data need to be reprocessed, and then "blended" back in to the data set"

   3. (dbs)Need to track event processing failures in dbs, as reported by EDM in the job report run, event, reason for failure.

   4. (dbs)use pass ID to determine what data is interesting to analyze. Pass ID is equivalent to processing path name.

   5. (dbs)Allow for fruit salad mixing of data in a processed dataset.

   6. (dbs)Store filter information for each file (stream) for multi-output skimming

   7. (dbs, need more info)support for pileup input datasets/files.

   8. (use case-?)data quality information. What level of granularity? run, luminosity block, event?

   9. (dbs, use case-?)Must be able to track luminosity for analysis dataset. Run over N luminosity blocks, some fail, some pass. How to integrate luminosity for all data used.

  10. (dbs)All copies of a physical file are lost. Must be able to mark the status as unavailable or lost. Needs to be reported to DLS, and PhEDEx also.

  11. (ReqSys?)Need a way to specify and submit processing requests for MC generation, and general data processing.

  12. (ProdAgent)Recovery of files that were not processed due to input failures, or high error rates in application.

  13. (ProdAgent)Recovery of data lost after processed. Outfile is lost 1. before entered into DBS, or 2. after entered in DBS.

  14. (dbs) Users need to store special data, such as log files or personal or group data.

  15. (dbs)Group needs a way to specify "official" data sets or analysis datasets.

  16. (dbs)Primary datasets must be special and have special meta data or ability to link to other information that characterizes it.

  17. (review w/ ProdAgent, dbs)Merge and remapping requirements and use case. Is it really required (or advisable) that we get rid of intermediate files? Refer to SAM experience reported by Adam.

  18. (dbs)Authentication and authorization for DBS API

  19. (dbs)Who or which group created a particular set of data?

  20. (?)Naming convention for processed dataset. Can this be automated.

  21. (dbs, in progress)Schema review items: does family have any meaning in CMS? , Dans talk, review SAM schema, others.

  22. (use case-Drew,Lothar)Additions needed for CSA06: 1. find data based on application version, 2. link with DLS to find where data is.

  23. (use case) Need to go through analysis workflow (use cases) and determine additional API calls that will be needed.

  24. (use case-C.Jones)How is Analysis data set defined and how will it be used? Will it be static or dynamic (changes as new data is added or old removed)? Is there something like a snapshot? Are names prescribed for Anal DS, are they pre-fabed by physics groups or other admin policy. Can users create their own at will? What will UI look like, what is API required.?

  25. (dbs, FJR)Need way to have the set of tags (one for each subsystem) specifying calibration and alignment into DBS so it can be query-able. (it is in the config file)

  26. (dbs) Introducing data from out side official processing machinery, so-called offroad data"

  27. (usecase-?)Private and semi-private data sharing. How can you share data with colleagues at other institutions before it is ready for global scope publication.

  28. (dbs)Getting runs into DBS, need DBS to have relation to luminosity block information.

  29. (?)Need to have LFN naming convention.

  30. (use case, need more input) High level trigger information in DBS based on trigger mask, or trigger object, so can discover data based on trigger.

  31. (use case, need more input)Luminosity information stored in DBS? This information may change with time. Could have lumi block ranges and get luminosity from lumi DB.

  32. (dbs,FJR) Data tier is defined well enough to know what objects are in it. Some concern that this will change. How can we store the branch information from the FJR in a useful way? How will it be queried?

  33. (dbs) it is not correct to define parentage between processed datasets as some child data may be missing. Better to define at the file level.

  34. (dbs, some assumptions to clear up)Need way to select individual events from a particular tier given run/event.

  35. (ProdAgent, dbs)Need way to track, monitor, record usage patterns of the data access. Way to study history of data access.

  36. (dbs, need more input) How is file block name handled when published local to global scope DBS. Are files within the file block rearranged. Are more block attributes needed?

  37. (dbs,FJR)The FJR should contain as much as possible of the information that is recorded in DBS.

  38. (use case-?)What is the software environment an analysis user will be using while in the local scope? Who moves data and DBS info about the data from local to public scope and how is it accomplished? What is the software setup, configuration, and what tools are needed? What authentication and authorization is needed for this?

  39. (ReqSys?-Giulio, dbs-fnal)Use cases for the data processing request system for both MC and real physics data processing. What is the overlap between the request system and DBS? DBS and the Request system share discovery of 'applications' (i.e. the configuration). It is assumed this sharing will not be needed for CSA06 but should be understood for the final system.

  40. (ReqSys?-Giulio) Confusion about meaning of payload and workflow in MC request system [Guillio needs to make a glossary that synchs up with the DBS jargon].

  41. (ReqSys?, dbs)The Request System and DBS are related but Peter thinks should be 'decoupled'. It seems possible and useful to link up the parameter set discovery system of the two in the long term. Not needed for CSA06.

  42. (?) Explore use of defaults for data specification [e.g. do not make people say 'AOD' since that is the most requested]

  43. (need more inout)How is DBS used for Online? what might DBS Track?
         1. Laser data
         2. alignment stuff
         3. express stream
         4. dedicated calibration runs
         5. Physics data> ==> Does DBS begin before or after repacking? << big issue. 

  44. (use case) How is DBS used for Tier 0?
         1. Reconstruction
         2. Tracking calibration and alignment output data 

  45. (use case) How is DBS used in Tier1:
         1. Re-reconstruction
         2. skimming 

  46. (use case) How is DBS used in Tier 2?
         1. MC Production
         2. Data analysis 

  47. (usecase) How might DBS be employed for MTCC and Test Beam data?
         1. If a run table is added thenthis type of data could be added
         2. Under no pressure to make this work on a short time scale, but can use run for existing data to load the DBS catalog.
         3. Would be useful experience for the DBS system and good interaction form physisists. 

  48. (use case, dbs?) DBS/DLS/PheDex know about file/file block sizes
         1. How is this space 'assigned'/'allocated' at Tier-2 sites?
         2. Need tools to determine how disk (resources?) are used by "groups" in CMS?
         3. detail: Need to have concept of ownership (grolup/user) in either DBS or DLS
         4. research: map how people running jobs work within the constraints of the existing system (how do they deal with disk allocations, and stepping on one anothers resources?) 

49. (need to present to other cms db players - Lee) Concept map for DBS and how it relates to other event and non-event data repositories. (DLS/DBS, conditions DB, trigger DB, luminosity DB, et cetera.
  
50. (use cases) Data Discovery use cases and what should be available for discovery (Need subgroup to further explore this):
         1. detector experts looking at detector problems, Range of data (run range) with certain conditions, bad detector component, special trigger.
         2. What Physics meta data is needed to be related to datasets? Is it good enough to have 'simple' trigger name or all details of trigger needed? What high-level queryable things needed to store to find data? We need to make 'synthesized' easy to query versions AND allow full physics physics metadata discovery. 

  51. (dbs) From schema discussion:
         1. Three main concepts (more when have pictures of notes):
               1. Files
               2. Processings
               3. Datasets 
         2. We see no need to continue the concept of event collections.
         3. Schema was discussed and sketched out at workshop. Anzar will enter and draw using druid. 

\section{Use Cases}
\subsection{Use cases for CSA06}

 
\begin{enumerate}
\item{Physicist Tier 2 Creates Skim of Re-RECO at Tier 1}

Physicist polls list of available datasets for those produces by CSA06. He sees a job which he identifies as a skim by its name. Someone tells this physicist what Tier 1 has the data. The tool shows him the dataset identifier. He puts that identifier into a Prodagent job, along with the name of the Tier 1 on which to do processing.

\item{Physicist Tier 2 Discovers Tier 1 Data Available for Processing}

Physicist polls a list of available datasets for datasets produced by CSA06. The physicist recognizes Re-RECO data because of the dataset name. The program (or web page) provides a dataset identifier.

\item{Physicist Tier 2 runs analysis job}

Physicist asks CRAB to run an analysis job.

\item{Alignment Person Recalibrates using Z to mu-mu to produce a new version of alignment}

\item{Physicist polls a tool which shows a list of available datasets.} He identifies CSA06 jobs by name. [Should be done at CAF, but not for CSA06.]

He submits the alignment job to CRAB or Prodagent, which returns a new version of the alignment. Alignment person puts this alignment into alignment database.

\item{Production Manager Tier 2 Re-runs RECO analysis}

Alignment person tells production manager about newly-available calibration. Production manager specifies new alignment in Prodagent or CRAB configuration file. Production Manager specifies a subset of the RAW dataset and submits it to run. Prodagent puts result into global DBS. (It could be a job robot that runs this instead of Prodagent or CRAB. How do you insert new detector conditions? Do you need a new version of the software or configuration.)

\item{Production Manager Watches CSA06 Progress}

Production Manager wants to see what data exists and what file blocks are transferred. They open a web page tailored to CSA06. It queries the DBS for file existence and PhEDEx for file transfer rate. (Can PhEDEx tell you this? Would something else do it, like the DLS?)

\item{Production Manager Creates Real HLT Data from MC Data}

Production Manager runs the HLT on a large dataset of MC Data. Production Manager uses Y to put that data into the DBS. Only five million events go through the HLT from all of the MC data, and DBS tracks these.

\item{Tony at Tier 0 Puts Real Data into Global Scope}

He creates data with a set of scripts. He asks X to make file blocks from that data. He asks Y to put that data into the DBS.

\item{Physicist Tier 2 Runs Skim on One of Six Copies of AOD}

Physicist sees in exploration tool that AOD is available at Tier 1 and can tell that this is AOD from Re-RECO. Physicist asks X where this data is stored. Physicist puts into the CRAB configuration the name of the dataset and where to do the computation. CRAB puts the resulting file into the DBS. 

\end{enumerate}
\subsection{Use Cases Long-term Project}
\subsubsection{Monte Carlo Production}
\subsubsection{Physics Data Production}
\subsubsection{Data Analysis}

\section{DBS client server Archetecture}
\subsection{Requirements}

+ DBS Access What are the avenues through which the DBS information will be accessed?

\subsubsection{Overview of usage}

   1. Production processing
         1. ProdAgent 
   2. Group processing
   3. Individual processing
   4. Individual and group analysis
         1. CMSSW
         2. ROOT
         3. Other tools, shells, et cetera 
   5. Interactions and relationships with other CMS databases
         1. Trigger, Luminosity
         2. Online Runs
         3. Conditions, et cetera. 

\subsubsection{Access}

   1. API
         1. CRAB
         2. ProdAgent
         3. Framework (CMSSW) through some wrappers?
         4. ROOT ?
         5. Primary language is expected to be Python, but could include C++, Java, et cetera. 
   2. Command Line Input
         1. Designed to work conveniently with standard Unix commands
         2. based on API 
   3. Web tools
         1. Initial tools will use API
         2. May grow beyond API
         3. Include high level search and discovery features 

\subsection{Options for service architecture technology approaches}
There are at least two issues here, 1. server technology, and 2. transport and messaging layer. The two can be closely related as the tools provided for the messaging layer are better for some server solutions than others.
Server options

\begin{enumerate}
\item   1. CGI/Perl: Continue to develop CGI w/ Perl scripts
\item   2. CGI/Python: Continue to use CGI, but rewrite scripts in Python
\item  3. Servlets: Use Tomcat servlet container and implement API in Java servelets.
\item   4. Commercial Appserver: Use a commercial application server solution, for example JBOSS and Hibernate.
\item   5. C++ DBS AppSrv?: Reuse the DBS App server developed at FNAL under a standard server technology, like Apache, or Tomcat.
\item   6. RAD or Agile Web technology: Use one of the recent rapid web development platforms, for example Ruby On Rails (Ruby), or Django (Python), Zope 3. 
\end{enumerate}
Transport and messaging layer
\begin{enumerate}
 \item  1. Assume the transport layer is HTTP
 \item  2. Messaging layer can have a couple of choices
         1. Standard like SOAP, XML-RPC, REST
         2. We make something that is performant (like the XML messages used now for CGI, or something similar to Frontier). 
\end{enumerate}
\subsubsection{Criteria for decision}
\begin{enumerate}
 \item   1. Must be proven technology
  \item  2. Not out dated
  \item  3. Well supported by existing tools and documentation
 \item   4. Flexible framework that we can grow to meet future needs
 \item   5. Provide needed scalability
  \item  6. Easily maintainable
 \item   7. Accommodate schema evolution
 \item   8. Accommodate API evolution
 \item   9. Server functions must be well factorable
 \item  10. Provide "local" and "global" scope functionality. This implies a. using multiple database technologies and a. having access through local library or remote server.
 \item  11. Meet performance needs. This depends on transport and messaging layer employed.
\item   12. Web Services support. This is a desirable, and determines the flexibility for the future. 
\end{enumerate}



Decision matris can be found in Table~\ref{tab:ecal-full-read}.
\begin{table}[htb]
    \caption{Features provided by each technology option. X=Yes,*=Maybe, but involves more work, ?=Unknown, or too soon to tell, Blank=No  }
    \label{tab:tech-options}
    \begin{center}
      \begin{tabular}{|l|c|c|c|c|c|} \hline 
Criteria/Approach & CGI/Perl & CGI/Python & Java Servlets & Jboss &Ruby\\
 Proven           & X & X & X & X &      \\
 Modern           &   &   & X & X &    X  \\
 Supported        & X & X & X & X &    ?  \\
 Flexible         &   &   & X & X &    ? \\
 Scalable         & X & X & X & X & ?    \\
 Maintainable     & * & X & X & X &    ?  \\
 Schema evolution & * & * & X & X &    ?  \\
 API evolution    & * & * & X & X &   ?  \\
 Easily factorized& * & * & X & X &   ?  \\
 Mult DB backends &   &   & X & X &    ?  \\
 Accessed via lib 
     or server    & X & X & X & X &    ? \\
 Performance      & X & ? & X & X & *       \\
 WS Support       &   &   & X & X & ?       \\ \hline
      \end{tabular}
    \end{center}
  \end{table}  
%put into tablualr format
%\begin{verbatim}
%Criteria\Approach 	CGI/Perl 	CGI/Python 	Java Servlets/tomcat 	JBOSS/Hibernate 	C++ DBS AppSrv? 	RUBY on Rails
%1. Proven technology 	X 	X 	X 	X 	  	 
%2. Not out dated 	  	  	X 	X 	  	X
%3. Well supported 	X 	X 	X 	X 	  	?
%4. Flexible framework 	  	  	X 	X 	  	?
%5. Scalability 	X 	X 	X 	X 	X 	?
%6. Maintainable 	* 	X 	X 	X 	  	?
%7. Schema evolution 	* 	* 	X 	X 	X 	?
%8. API evolution 	* 	* 	X 	X 	X 	?
%9. Well factorized 	* 	* 	X 	X 	X 	?
%10a. Mult DB backends 	  	  	X 	X 	X 	 
%10b. Access via lib or server 	X 	X 	X 	X 	X 	 
%11. Performance 	X 	* 	X 	X 	* 	?
%12. WS Support 	  	  	X 	X 	* 	?

%    * X=Yes
%    * *=Maybe, but involves more work.
%    * ?=Unknown, or too soon to tell
%    * Blank=No 
%\end{verbatim}

\section{Data discovery needs and options}


\section{Framework needs}

\subsection{CMSSW Framework provenance/DBS interaction}

The "Unambiguous identification of reconstruction results" section of the "CMS Core Software Re-engineering Roadmap" is mostly correct (a few things are not implemented yet, and some recent design decisions are not reflected there yet). This page provides a brief summary of the the aspects of the framework provenance which may have implications for the DBS provenance database.

\subsubsection{Framework provenance summary}

There are two kinds of provenance produced by the framework: (1) module configuration for the process; (2) event-by-event, for each object produced, what event objects were accessed by the module producing that object. Event-by-event provenance is dropped if the objects are dropped (TODO: verify); module configuration is always stored (it was decided that garbage-collecting unreferenced configurations was not worth the effort) (TODO: verify). Dropping the event-by-event provenance can lose processing history, since only direct parentage is kept (TODO; verify), but the sequence of processing steps is stored. The module configuration stored is currently the configuration for the entire process (i.e. the union of all the module configurations) (TODO: verify), not a module-by-module configuration, but the module by module configuration can be recovered from the overall process configuration. Provenance tracking for the EventSetup is planned, but not (as of May 2006) implemented yet. The full provenance is written to the output file.

Names/paths of files containing event data are untracked parameters, so they are not recorded in the provenance. There are also no file unique identifiers--the framework, except for the input and output modules, deals with events as objects with no association with any file. Filenames of some conditions data are tracked parameters.

Each CMSSW production process has a "job configuration" name. The job configuration name is a component of the branch name for objects produced in that job, so branch names will be different for objects where the processing steps differ (e.g., SIM/DIGI/RECO in one job vs. three separate jobs). The framework partially hides this, but it is visible in ROOT, so there is a strong incentive to use an identical sequence of processing steps for all objects of a given type. As of May 2006, there are proposals for aliasing branch names which may change this consideration if every branch has an explicitly configured alias.

EDM fast merge will, in one job, only process files with identical provenannces. Prior to the 0.8.0 release, the CMSSW Framework had the same restriction. Starting with 0.8.0, CMSSW plans to support "fruit salad" processing, where files with different provenance may be processed in the same job if the provenances are compatible. Provenances are considered compatible if the object name to number mappings are the same for all objects in common between the files. The fast merge does not modify any objects, and does not create any provenance--so far as CMSSW is concerned, fastmerge is completely transparent.

\subsubsection{DBS implications}

At the file level of the DBS, the CMSSW provenance mostly provides constraints. File parentage must be tracked entirely by the DBS--CMSSW does not track it. CMSSW provenance provides a processing history at the object level, not the file level, and the object history may not be a complete record of the processing steps that produced the file containing that object if intermediate objects have been dropped from the output.

The framework allows objects of the same type but different provenance if they differ in the process name, process instance label, or module label. The framework provenance may also include module configurations for objects which are not in the output file, since module configurations are not "garbage collected". The presence of multiple configurations for the same module--for objects which may not even be in the corresponding files--will limit the accuracy and reliability of queries on any "file level" provenance summary. A simple example of this would be if the HLT places reconstructed objects in the output stream which are also produced in prompt RECO or re-reconstruction. At the very least, this would mean that any query on the DBS provenance must include the process name as part of the query, and that may not be sufficient for the analysis use cases envisioned.

\subsubsection{Notes and links}

    * How to Configure Output Modules documents how branch names are constructed
    * EDM Paths and Trigger Bits explains how to selectively write events based on "trigger" paths
    * The framework currently reads only one data file at a time--there is no ability to join data from multiple files in the baseline. Adding this ability is (as of June 2006) under discussion. 
\section{Schema Description and Design}

\begin{enumerate}
\item{File}
 A File can contain multiple runs. A run can contain multiple files. The
 event range in a file represents the lowest bound event number to the
 highest bound event number. For example in a run r with a File f has
 event range 19,130. This does not imply that all the events are
 contiguous . There might be some events that can be missing. For example
 in this file f , only events 19,20,25,130 are present and rest of then
 are bad or missing.
 
 \item{FileRun}
 This table will represent many to many relationship in file and run.
 Note that the event range is an attribute of the FileRun table and not
 just of the File Table. This is required when multiple files will be
 merged into one single file. Then a File can spawn multiple runs and can
 spawn multiple event ranges. To discover all the event in this file we
 need to make event range an attribute of FileRun table. For example say
 a file f1 in run r1 contains events 3,16 and file f2 in run r2 contains
 events 5,15 and file f3 in run r1 contains events 2,13 . After we
 can merge f1,f2 and f3 into a big file ff, now FileRun table would
 contain two entries for file ff and run r1 and run r2. First entry for
 ff,r1 would have event range 2,16 and another entry for ff,r2 would
 contain 5,15. Note that of the files to be merged are in the same run
 then the event range would be minimum of lower bounds to maximum of
 higher bounds min(3,2),max(16,13).
 
 \item{FileParentage}
 This represents the lineage of the file. All the parents of the file are
 represented in this table. This file parentage is required at the file
 granularity. If it is not presents at this granularity , then there
 would be no way to correctly determining the exact parent of any
 particular file. Please note that this parentage is also specified at
 the ProcessedDataset level. The only reason for that is the optimization
 for discovery. One can easily find the parents of a dataset via
 traversing all the files and finding the parents of those files and then
 locating the datasets those files exists in. To optimize this discovery
 we need parentage at the dataset level too.
 
\item{ Block}
 This entity is entirely a concept of size and number of files. It can
 contain many files but if the collective size of those files reaches a
 certain limit (can be set by DBS admins), then it cannot accommodate any
 more files in it. In that case a block can be closed and a new block can
 be opened for adding new files into it. A block can be uniquely
 determined by its name that can also include dataset path along with its
 id.

(Dan's comments)
As was pointed out to me at the workshop, a block is supposed to
be homogeneous--all the contents should of a block should be of
the same data tier and processed data set.  This is important for
PhEDEx and for some of the management data discovery use cases we
discussed at last Friday's meeting.  PhEDEx in particular needs to
be able to efficiently find all the blocks for a given data tier
and processed data set, so this constraint may need to appear in
the schema as an optimization.
 
 \item{Lumi}
 A file can contains many Lumi Sections and a Lumi Sections can be
 contained in many files. These Lumi Sections are used for describing an
 analysis dataset which is conceptually more or less not related with
 files.

(Dan's comments) Missing here is what a luminosity section is.  The definition can be
found on this page:

https://twiki.cern.ch/twiki/bin/view/CMS/CMSTORTagonline

What's relevant for the DBS is that a luminosity section maps
onto a (contiguous) set of events defined at the time the data
are taken.  Since the the luminosity section boundaries are
defined by the DAQ, they will never change.  Luminosity sections
are not supposed to cross file boundaries--when the framework
splits a file for size limit reasons, it is supposed to do so
at a luminosity section boundary (I don't believe this is
implemented yet).

\item{DataTier}
 DataTier is infact related with the AppConfig via ParameterSet. But we
 need DataTier linked with Files also , since a physicists would require
 to know all the file in a dataset which has a certain DataTier. If
 DataTier would just be an ProcessedDataset Attribute, then the above
 query could not be executed. This is because a ProcessedDataset can
 spawn multiple DataTiers. If there is a restriction that a
 ProcessedDataset would only have a single DataTier then we would not
 required DataTier linkage with files at all. Also a File can have
 muliple DataTiers also, that iss why all the more reasons to link
 DataTier with File. Note that FileTier is enough for associating and
 determining the DataTier with in a file as well as within a
 ProcessedDataset. So why would we need linkage of DataTier with a
 ProcessedDataset. The reason is same as that of File Parentage and
 ProcessedDataset Parentage. This is just an optimization for determining
 the DataTier with in a DataSet. One has to traverse all the files
 otherwise to know the DataTiers.

(Dan's comments)

As you note, a single AppConfig can write multiple files of different
data tiers at the same time, so the AppConfig <-> data tier
relationship is ambiguous.  The new definition of data tier is that a
data tier comprises a list of objects defined by a release which are
written to files of that tier.  The objects written to a file of a
given data tier will be defined by a release-dependent configuration
fragment, so what objects go into an AOD or RECO file may vary from
release to release.  I don't know how practical it will be to retrieve
these lists from the full ParameterSet -- it may be desirable to have
a separate table mapping (data tier, release) to a list of objects.
 
 \item{PrimaryDataset}
 PrimaryDataset is a placeholder for any type of data like raw or
 production. It is determined by the name and the description. Now this
 description can be of 3 different types: Monte Carlo type, Trigger Path
 type or some Generic Description. Therefore there are 3 tables
 representing each one. PrimaryDataset can contains raw data which means
 that it was not processed. To manage this within the schema, one can
 create a dummy ProcessedDataset with no link to any Application
 (AppConfig in this case). All the raw files can now be placed in this
 dummy ProcessedDataset

(Dan's comments)

There's actually two kinds of raw data: DAQ-RAW is the raw response
data from the detector in FED format, which is input to the HLT; RAW
is the output of the HLT.  Normally what will be injected into the DBS
is RAW data from the HLT, and the HLT is a cmsRun application with a
standard parameter set, so the normal RAW data will have real
processed data sets, not a dummy one.

 
 \item{Application}
 Application is uniquely determined by the Application Version,
 Application Executable and Application Family. There are three table
 representing each one and collectively they represents an Application
 \item{ParameterSet}
 ParameterSet is a collection of Parameters which are used to make a
 process. An entry in the ParameterSet can represent a single parameter
 or a composite parameter. For example a single parameter would contain
 name, value ,type and hash and a composite parameter would contain
 name, value ,type and hash  with or without content. Also a composite
 parameter would contain lot of other single or composite parameters.
 This is represented by the ParameterBinding table. One can look at these
 as a hierarchical directory structure. A single parameter would be
 lowest level directory which does not contain any other directory. A
 composite parameter would be a directory that may or may not contain
 other directory and may or may not contain a file (content). Hash
 uniquely identifies a parameter (both single and composite)
 
(Dan's comments)

I believe there's a way of getting the framework's pset parser to
give access to its internal parse tree, which could be used to
construct the QueryableParameterSet.  Parameter set's can have
fairly complicated nesting, so I'm a little worried about what a
query on this will look like, but I guess this is a reasonable
starting point.

 \item{AppConfig}
 A unique combination of an Application and a ParameterSet would
 represent an AppConfig. A file will be produced by a single AppConfig.
 That is why we have many to one relationship between AppConfig and File
 table. Note that a File has a relationship with ProcessedDataset, but
 that is not enough to determine the exact AppConfig used to produce the
 file. The reason being that a ProcessedDataset can spawn multiple
 AppConfig.
 
(Dan's comments)

While this is true, keep in mind that the framework can now merge
files with different provenance (the "fruit salad" mode), so different
objects in a single file may have different parentage/processing
history.

\item{ ProcessedDataset}
 Is uniquely represented by the PrimaryDataset , AppConfig and the input
 ProcessedDataset if any. Also, a ProcessedDataset can contain various
 AppConfig (Different version of Application). This satusfy the use case
 of creating many files with same application but different version of
 the software used. It can contain multiple files and therefore has
 one to may relationship with file. Also it can contain multiple DataTier
 which itself can spawn multiple ProcessedDataset and therefore many to
 many relationship among them. The reason for this linkage is for
 optimization purposes only and is described in FileTier section
 
 \item{AnalysisDataset}
 It is uniquely determined by Lumi Sections and ProcessedDataset. An
 AnalysisDataset is a subset of ProcessedDataset. From one
 ProcessedDataset many AnalysisDataset can be derived. Driving a new
 AnalysisDataset from an existing AnalysisDataset would mean to create a
 new ProcessedDataset. So essentially it will be a new AnalysisDataset
 from a ProcessedDataset again. Also AnalysisDataset can contain many
 Lumi Sections and same Lumi Sections can spawn multiple AnalysisDataset.
 Therefore we have AnalysisDSLumi table which represents many to many
 relationship among these entities.

(Dan's comments)

Analysis data set is more complicated than this.  The definition we've
been using is "a list of luminosity sections which were specified by
running an analysis query on a processed dataset at a particular
instant of time".  A processed data set may contain multiple instances
of a luminosity section processed with different application
configurations, so an analysis data set is determined by a processed
data set, a list of luminosity sections, and an AppConfig for each
luminosity section.

We haven't discussed how (or if) this maps onto MC data.

\section{Client API}
\section{Plan of Work}

\begin{thebibliography}{9}
  \bibitem {dar} {All bogus entries below}
  \bibitem {smtp} {Simple Network Management Protocal}
  \bibitem {mrtg} {Multi Router Traffic Grapher}
  \bibitem {pool-ral} {POOL-RAL: POOL Persistence Framework, url{http://pool.cern.ch/ .}
  \bibitem {frontier}{FroNTier Project, see S. Kosyakov, et. al.,''Frontier: High Performance Database Access Using Standard Web Components,'' Proceedings of the CHEP\,04 Conference, Interlaken Switzerland, 27 September - 1 October, 2004,
Available at http://lynx.fnal.gov/ntier\-wiki/Additional\_20Documentation.}
  \bibitem {tomcat}{Tomcat home page: http://tomcat.apache.org}
  \bibitem {squid}{D. Wessels, Squid: The Definitive Guide, O'Reilly and Associates, 2004. Squid home page: http://www.squid-cache.org.}
  \bibitem {squid-accel-mode}{op. cite~\cite{squid}, pp. 302-314.}}
%last } is a mystry but latex asked for it. 
  \bibitem {squid-objects}{For additional information see http://lynx.fnal.gov/ntier-wiki/Squid\_20performance\_20for\_20big\_20objects}
\end{thebibliography}

\pagebreak



\end{document}